\documentclass[12pt]{report}
\usepackage[margin=1in]{geometry}

\usepackage{amzmath}

\usepackage{amsthm} % Theorems
\usepackage{lipsum} % Filler text
\usepackage{fancyvrb}

\begin{document}
    \chapter{Chapter title}

    \lipsum[1]

    \begin{dfnbox}{asdf}{asdf}
        \dfntxt{asdf} is the word you get when you press the first four home-row keys from left to right.
    \end{dfnbox}

    \begin{thmbox}{Cool Theorem}{cool}
        This \LaTeX\ template is cool.
        \tcblower
        \begin{proof}
            \lipsum[1]
        \end{proof}
    \end{thmbox}

    As seen in \nameref{thm:cool} (Theorem \ref{thm:cool}), this template is pretty cool.

    \begin{exbox}{Template}{}
        \lipsum[1]
    \end{exbox}

    \begin{tecbox}{Using This Template}{}
        To use this template, simply download the \href{https://raw.githubusercontent.com/alexmingzhang/latex-notes-template/main/amzmath.sty}{amzmath.sty} file, and include it using the command \verb|\usepackage{amzmath}|.
    \end{tecbox}

    \begin{codebox}{Cool Code}{cool}
        \begin{amzcode}{c++}
            #include <iostream>

            int main() {
                std::cout << "Wassup" << std::endl;
                return 0;
            }
        \end{amzcode}
    \end{codebox}

    The \nameref{code:cool} is pretty cool!

    \chapter{Really really long title that has to use two separate lines}

    \makeamzindex

\end{document}
